\documentclass[a4paper]{article}

\usepackage[]{fullpage}

\usepackage[]{amsmath}

% This statement puts a one-line spacing between two adjacent paragraphs
\setlength\parskip{\medskipamount}

% This statement cancels the indentation of the paragraph's first line
\setlength\parindent{0pt}

\begin{document}

{\Huge Proof of a Riddle}


{\LARGE Written by: Shlomi Fish}


\section{The Problem}

We need to prove that for every natural number $n > 0$, there exists a
decimal number of $n$ digits, which can be wholly divided by $5^n$, and
all of its digits are odd.

\section{Methodology}

We will prove a stronger claim. We will demonstrate that if $b_n$ is the
corresponding number for $n$, then it can serve as a suffix for $b_{n+1}$
, by adding another most significant digit.

More formally:

\begin{enumerate}
\item $b_1$ = 5.
\item For every $n$, there exists an $a \in \{1,3,5,7,9\}$ so that
$b_{n+1} = b_n + a \cdot 10^n$ and $b_{n+1} \mod 5^{n+1} = 0 $.
\end{enumerate}

\section{Proof}

The proof would be by induction.

\subsection{Induction Base}

It holds for $n = 1$ as 5 is a one-digit number that is wholly divisable
by $5^1$.

\subsection{Induction Step}

Let's assume it holds for $n$ and show that it also holds for $n+1$.

Now:

\[b_{n+1} = b_n + a \cdot 10^n \]

According to the induction step $b_n$ is wholly divisable by $5^n$ and so
is $10^n = 5^n \cdot 2^n$. So we can divide the expression by $5^n$ and
try to find an $a$ so that the quotient is divisable by 5. We get:

\[ b_{n}' + a \cdot 2^n \]

$b_{n}'$ has some modulo 5, and $2^n$ has a non-zero modulo. The values that
$a$ can assume (1,3,5,7,9) contain all the modulos of 5. Since 5 is prime
, and its modulos are a group, we can get all modulos by multiplying a
given non-zero modulo by the other modulos. So we can choose an $a$ so that
the expression modulo 5 evaluates to 0. Thus we can divide this $b_{n+1}$ by
$5^{n+1}$ as well.

Q.E.D.

\end{document}

